\section{Technology}
\setlength{\parindent}{10ex}
For this work I used Python 3.6.1 as the main programming language.
The Python community offers a multitude of data science libraries that I leveraged.
To name a few, numpy, pandas, and sklearn.
In short, Python was chosen for this project because it offers the best end to end tooling for this work.

\subsection{Sci Kit Learn}
SKLearn is the main \ac{ML} library used in this project.
The \ac{API} is simple and intuitive, and it also exposes the classes for new user classes to be created.
A variety of regression, classification, and ensembles are included in the library.
From this library this project used the following models: SVMRegression, NaiveBayesRegression, LinearRegression, KNearestNeighborsClassifier, RandomForestClassifier, DecisionTree, MLPClassifier, AdaBoostClassifier, GradientBoostClassifier, BaggingClassifier, and Voting Classifier.
The sklearn metrics library was leveraged to measure preformance results.
This includes functions for measureing \ac{RMSE}, Balanced Accuracy, F1 score, R Squared score, and Precision/Recall.
