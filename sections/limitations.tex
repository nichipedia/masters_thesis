\section{Future Work}
\setlength{\parindent}{10ex}
There are many interesting avenues to explore from this research.
A first avenue will be to perform the experiments at a higher spatial resolution.
This work used two-minute spatial grids for representing all features and bathymetry.
A two-minute grid was used for the memory and size advantages, allowing the computations and models to be run on a general-purpose work-station.
Modern datasets for bathymetry are often represented in higher resolution grids.
Naturally, the results will scale to a higher resolution.

\par
Another interesting avenue will be to explore selecting features based on geophysical location.
This could be tested by performing feature selection on the models across different coverages of the globe.
A simple genetic algorithm could be run for each model.
The resulting features will then be used for training and the best performing model will be selected with the optimum features.
This experiment will eliminate features that do not locally support the model while highlighting the locally important features.

\par
The same approach to feature selection could be taken to parameter tuning.
Running a similar genetic algorithm to tune optimum parameters could potentially yield better results.
Searching for optimal coverages is also has great potential.

\par 
Defining the best decision function for model selection is also a worthy investment.
The core experiment in this project used a naive spatial boundary for selecting models.
However, it is possible that geographic features were primary contributors to model success.
Identifying several model selection functions and finding an optimal function will allow for conclusions to be drawn in regards to model success.

\par
Finally, this work was performed using predicted bathymetry from the ETOPO dataset.
As already stated, this was done in order to prove the viability of the models, not to prove the accuracy could be greater than an \ac{EGM}.
However, this is an experiment that can be executed.
Training these models against true bathymetry and comparing the metrics to existing \ac{EGM}s will give an indication of the ability to predict bathymetry.