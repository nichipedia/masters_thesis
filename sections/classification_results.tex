\section{Classification Results}
\setlength{\parindent}{10ex}
%Maybe reword this opening chapter???
The next approach used in this work was to convert the regression problem into a ordinal classification problem.
Convertion was preformed by mapping the bathymetry values into dicrete classes.
This conversion is a natural step taken in response to the regression results.
While the regression models yielded a poor accuracy, they held a strong R2 score.
This score is indicative of correlations in the dataset.
These underlying correlations will benefit a classification model, as they are, simpler and easier to fit.
In addition, the continous nature of bathymetry allows this problem to easily be converted into a ordinal classes.

\par
The ordinal classes were created on a interval of 150m.
This was done to improve accuracy beyond the reported findings of \cite{jena2012prediction}.
Validation was preformed using a 10 fold cross validation using balanced accuracy as the scoring function.

\begin{center}
    \begin{table}[htb]
        \begin{tabular}{|c c c|}
            \hline
			\textbf{Model} & \textbf{Average F1 Score} & \textbf{Mean Balanced Accuracy} \\
			\hline
			Random Forest & 0.81 & 0.82 \\
            Bagging & 0.884 & 294.92m \\
            Decession Tree & 0.885 & 265.43m \\
			\hline
        \end{tabular}
        \label{table:CLASSIFICATION_RESULTS}
        \caption{Regression Results}
    \end{table}
\end{center}

\subsection{Classification Results Discussion}
The Random Forest model preformed excellently with a balanced accuracy of 82\%.
Breaking down the results by class the classifier predicted some classes with greater precission than others.
This is indicative that the model responded to certain trends in the data.
This was also observed in the Bagging and simple Decision Tree Models.