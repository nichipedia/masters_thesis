\section{Classification Results}
\setlength{\parindent}{10ex}
%Maybe reword this opening chapter???
The next approach used in this work was to convert the regression problem into an ordinal classification problem.
Conversion was executed by mapping the bathymetry values into discrete classes.
This conversion is a natural step taken in response to the regression results.
While the regression models yielded a poor accuracy, they held a strong R2 score.
This score is indicative of correlations in the dataset.
These underlying correlations will benefit a classification model as they are simpler and easier to fit.
In addition, the continuous ordered nature of bathymetry allows this problem to easily be converted into a ordinal classes.

\par
The ordinal classes were created on a interval of 150 meters.
This was done to improve accuracy beyond the reported findings of \cite{jena2012prediction}.
Validation was preformed using a 10 fold cross validation using balanced accuracy as the scoring function.

\begin{center}
    \begin{table}[htb]
        \begin{tabular}{|c c c|}
            \hline
			\textbf{Model} & \textbf{Average F1 Score} & \textbf{Mean Balanced Accuracy} \\
			\hline
			Random Forest & 0.81 & 0.82 \\
            Bagging & 0.80 & 0.79 \\
            Decision Tree & 0.44 & 0.47 \\
			\hline
        \end{tabular}
        \label{table:CLASSIFICATION_RESULTS}
        \caption{Regression Results}
    \end{table}
\end{center}

\subsection{Classification Results Discussion}
The Random Forest model performed excellently with a balanced accuracy of 82\%.
Breaking down the results by class, the classifier predicted some classes with greater precision than others.
This is indicative that the model responded to certain trends in the data.
These results show that a Random Forest classification model can predict 82\% of the Earth's bathymetry within 150 meters of error.

\par
The Bagging classifier performed on par with the Random Forest Classifier with a balanced accuracy of 79\%.
As seen with \ac{RFC}, bagging performed better at predicting some classes than others.
This supports the theory that trends in the data help certain classifiers predict areas better.

\par
The Decision Tree classifier performed significantly worse than the other models.
The 47\% balanced accuracy is not useable for predictions.
This model is not an ensemble, and leads to the hypothesis that a "weak" classifier is not ideal for predictions in this domain.