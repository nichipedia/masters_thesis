\section{Data Analysis}
\setlength{\parindent}{10ex}
The data for this paper was aggregated from many studies.
The ETOPO study was used for the bathymetry readings \cite{national1988etopo}.
Other features were aggregated from several studys.
See figure below for a list of features used and correponding origin study. 

\begin{table}[htb]
    \begin{tabular}{ |c|c| p{10cm}}
        \hline
            \textbf{Feature} & \textbf{Origin Study} \\
            Mantle Density & CRUST1 \cite{laske2013update} \\
            LAND One Hot & ETOPO \cite{national1988etopo} \\
            Crust Thickness & CRUST1 \cite{laske2013update} \\
            Low, Mid, High Crust Density & CRUST1 \cite{laske2013update} \\
            Estimated Current East, North, Mag & HYCOM \cite{chassignet2009us} \\
            Sea Nitrate, Phosphate, Salinity, Silicate Measurements & WOA \\
            Sea Temperate & WOA \\
            Sediment Thickness & CRUST1 \cite{laske2013update} \\
            BioMass Features & Goyetx \\
            Geoid Features & EGM \cite{pavlis2008earth} \\
            Wave height, period & WAVEWATCH \cite{tolman20072007} \\
        \hline
    \end{tabular}
\end{table}

\subsection{Feature Selection}
A gentic algorithm was used for feature selection \cite{yang1998feature}.
The initial population was created using one hots to represent the space of features.
A random number generator assigned each member a set of the potential features. 
Each member was then used to train a model and the resulting balanced accuracy of that model represented a fitness level.
The top two percent are then moved to the next generation.
While the rest of the top forty percent are randomly paried to produce offspring.
A cross over mutation is applied using the multi point cross over method.
A mutation is then applied to one percent of the population.

\par
This algorithm terminates when the balanced accuracy of a model exceeds a threshold.
The highest ranked member of the population is then chosen as the selected features.
The result of my implementation of this algorithm are documented in figure...

\subsection{Data Representation}
The data used in this project is organized into cell centered grids.
Where each grid cell value represents a average value across the cell.
Each grid represents a cell that maps to the EPSG:3857 coordinate reference system.
% this sentence below seems weird but meh?
The grids have a resolution which defines the number of cells along an axis.
High resolutions grids provide a more accurate coverage at the cost of memory.

\par
Data used in this project is fit into two minute bathymetry grid. 
This was done to minimize the interpolation required to fit higher resolution grids.
While also maximizing the original studies accuracies and avalaible resources.
However, some studies offer data at a smaller resolution than what is used in this project.
See the SRMTM30 \cite{becker2009global} for an example of such a study.

\subsection{Bathymetry Values}
The bathymetry data used for training are extracted from the ETOPO study \cite{national1988etopo}.
This data is a aggregation of predictions from gravitational models \cite{smith1997global} \cite{smith1994bathymetric}, and Multi Beam Echo Sonar (MBES) readings \cite{farr1980multibeam}.
MBES readings are accurate and reliable. 
Naturaly, it is cost and time prohibitive for a set of ships to survey the entire world.
The ETOPO dataset uses predicted values from gravitational models to fill the coverage.

\par
The bathymetry used for training was binned into classes for classification.
These classes where partioned on a interval of 150 meters.
This partitioning scheme was chosen to improve upon the results from a similar project \cite{national1988etopo}.


