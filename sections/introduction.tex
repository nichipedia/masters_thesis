
%Here is what I am about to tell you in this paper. Fairly informal and loose.
\section{Introduction}
\setlength{\parindent}{10ex}
The most accurate world bathymetry mappings come from a aggregate of predicted and measured sources. 
Earth Gravitational Models (EGM) are popularly used for predicting bathymetry \cite{becker2009global}\cite{smith1994bathymetric}\cite{smith1997global}\cite{smith2010planning}.
While sonar platforms such as the Multi Beam Echo Sonar (MBES) \cite{farr1980multibeam} are used for accurate measurements of the bathymetry.
Naturaly, it is not cost and time effective to collect large swaths of MBES bathymetry measurements.
EGMs use sattelite altimeter data to create gravitational models for predicting geoids.
In general, these models predict bathymetry with a error of 190 meters \cite{jena2012prediction}.

\par
There has been much work on using sattelite altimeter derived gravity to predict bathymetry.
This work focuses on using ocean features derived from other studies as predictors.
The goal being to create a model that can predict bathymetry using other features than gravity.
In this work I introduce an approach for optimizing model selection, feature selection, and a novel approach to predicting bathymetry.